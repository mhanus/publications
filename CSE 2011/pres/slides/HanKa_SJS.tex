\begin{frame}[t]
  \frametitle{HanKa -- qualification}
  \begin{itemize}
	  \item Comparison with real data obtained from NPP Temelin
   	\item HanKa's neutronics coupled with legacy T/H code of \\{\v S}koda Nuclear Machinery
  	\item<2-> Homogenization / reconstruction procedures neccessary
	  \item<3-> 3D eigenvalue problem
  \end{itemize}
  \begin{tikzpicture}[overlay]
    \only<2>{
      \node [xshift=2cm,yshift=-1.5cm] (hetero) {\includegraphics[scale=.25]{images/HanKa/avflux_proudyT.png}};
      \node [right of=hetero, node distance=5cm] {\includegraphics[scale=.25]{images/HanKa/heteroflux_proudyT.png}};
    }
    \only<3>{
      \node [xshift=5cm,yshift=-1.95cm] {\includegraphics[scale=.33]{images/HanKa/coreT.png}};
    }
  \end{tikzpicture}
\end{frame}

\begin{frame}[t]
  \frametitle{HanKa -- qualification}
  \vspace*{-.25cm}
  \centering\textcolor{structure.bg!95!blue}{\small Radial profile of reconstr. thermal ($g=2$) flux in all 48 layers}\\[-1.5em]
  \begin{tikzpicture}[overlay]
    \node [xshift=5.75cm,yshift=-3.75cm] (g1) {\includegraphics[scale=.33]{images/HanKa/our_vs_SJSg2T.png}};
  \end{tikzpicture}
\end{frame}

\begin{frame}[t]
  \frametitle{HanKa -- qualification}
  \vspace*{-.25cm}
  \centering\textcolor{structure.bg!95!blue}{\small Detail}\\[-1.5em]
  \begin{tikzpicture}[overlay]
    \node [xshift=5.25cm,yshift=-3.75cm] (g1) {\includegraphics[scale=.35]{images/HanKa/our_vs_SJSg2_vyrezT.png}};
  \end{tikzpicture}
\end{frame}

