\begin{frame}[t]
  \frametitle{Adaptive eigenvalue calculations}
  \vspace*{.85cm}
	\begin{itemize}
	  \item Two possible strategies:
	    \begin{enumerate}
	    	\item Outer source iteration \structure{+} inner mesh refinement iterations \
	    	  \only<2->{
	    	    \begin{itemize}
	    	     	\item related to the time-dep. problem of approaching to a steady state
	    	     	\item derefinement would be needed due to inaccurate initial source terms (due to inaccurate $\keff$)
	    	     \end{itemize} 
	    	   }         
	    	\item Outer mesh refinement iteration \structure{+} inner source iterations
	    	  \only<2->{
	    	    \begin{itemize}
  	    	    \item easier implementation
	    	      \item amenable for traditional source iteration acceleration methods
		          \item more reasonable for goal-oriented adaptivity aimed at the physically interesting quantities ($\keff$, assembly powers)
	    	     \end{itemize} 
	    	   }
	    \end{enumerate}
	\end{itemize}
	\alt<1>{\vspace*{3.35cm}}{\vspace*{.9cm}}\color{Black}\hrule width 5.1cm
	\begin{thebibliography}{\textwidth} 
	  \beamertemplatearticlebibitems
		\bibitem[Wang]{Wang}
		\footnotesize{Yaqi Wang},
		\newblock \scriptsize{HP-mesh adaptation for 1-D multigroup neutron diffusion problems}    
		\newblock \scriptsize{Master's thesis, Texas A\&M University, 2006}
	\end{thebibliography}
\end{frame}

\begin{frame}[c]
  \frametitle{\textit{"Outer-adaptation, inner-eigenvalue"}}
	% Define block styles
\tikzstyle{decision} = [diamond, draw, fill=blue!10, minimum width=2em, text width=2em,
    text badly centered, inner sep=0pt,drop shadow]
\tikzstyle{block} = [rectangle, draw, fill=blue!10, node distance=1.25cm,
    text centered, rounded corners, minimum height=1em,drop shadow]
\tikzstyle{line} = [draw, -latex']

% Define the arm and angle options
\def\myarm{1cm}
\def\myangle{0}
\tikzset{
  arm/.default=1cm,
  arm/.code={\def\myarm{#1}}, % store value in \myarm
  angle/.default=0,
  angle/.code={\def\myangle{#1}} % store value in \myangle
}

% Define the myncbar to path
\tikzset{
    myncbar/.style = {to path={
        % We need to calculate a couple of coordinates to help us draw
        % the path. 
        let
            % Same as (\tikztotarget)++(\myangle:\myarm)
            \p1=($(\tikztotarget)+(\myangle:\myarm)$)
        in
            -- ++(\myangle:\myarm) coordinate (tmp)
            % Find the projection of the (tmp) coordinate
            % on the line from the target to p1
            -- ($(\tikztotarget)!(tmp)!(\p1)$)
            -- (\tikztotarget)\tikztonodes
    }}
}

\pgfdeclarelayer{background}
\pgfsetlayers{background,main}    
\begin{tikzpicture}[node distance = .8cm, auto]
    % Place nodes
    \node [block] at ($(current page.north)+(-2.5,-2.45cm)$) (SI11) {\scriptsize $\MM\flfl_{hp}^{(n)} = \FF\flfl_{hp}^{(n-1)}\left/k_{\mathrm{eff},hp}^{(n-1)}\right.$};
    \coordinate (a) at ($(SI11.north)+(0,.85cm)$);
    \node [block, below of=SI11] (SI12) {\scriptsize $k_{\mathrm{eff},hp}^{(n)} = k_{\mathrm{eff},hp}^{(n-1)} \frac{\norm{\FF\flfl_{hp}^{(n)}}}{\norm{\FF\flfl_{hp}^{(n-1)}}}$};
    \node [decision, below of=SI12, node distance=1.65cm] (keff1) {\scriptsize $k_{\mathrm{eff},hp}^{(\!\infty\!)}$};    
   
    \node [block, below of=keff1, node distance=1.6cm] (ref_glob) {\scriptsize refine mesh globally};
    
    \node [block, right of=SI11, node distance=5.4cm] (SI21) {\scriptsize $\MM\flfl^{(n)} = \FF\flfl^{(n-1)}/\keff^{(n-1)}$};
    \coordinate (b) at ($(SI21.north)+(0,.6cm)$);
    \node [block, below of=SI21] (SI22) {\scriptsize $\keff^{(n)} = \keff^{(n-1)} \frac{\norm{\FF\flfl^{(n)}}}{\norm{\FF\flfl^{(n-1)}}}$};
    \node [decision, below of=SI22, node distance=1.5cm] (keff2) {\scriptsize $\keff^{(\!\infty\!)}$};
         
    \node [block, right of=ref_glob, node distance=5.4cm] (err) {\scriptsize global/local err. est. (GE/LE)};        
    \node [decision, below of=err, node distance=1.45cm] (mesh) {\scriptsize small GE?};

    \node [block, left of=mesh, node distance=4cm] (adapt) {\scriptsize refine mesh adaptively};
    \node (end) at ($(mesh.east)+(1,0cm)$) { ... };

    % Draw edges
    \path [line] (a) -- (SI11);
    \path [line] (SI11) -- (SI12);
    \path [line] (SI12) -- (keff1);
    \path [line] (keff1.east) -| ++(1.4,0cm) node [above,pos=.15] {\scriptsize no} node [pos=.3,yshift=-14pt] {\scriptsize $n=n+1$} |- ($(SI11.north)+(0,.2cm)$);
    \path [line] (keff1) -- node [near start] {\scriptsize yes}(ref_glob);
    
    \path [draw] (ref_glob.east) to[myncbar,angle=0,arm=1.0] (b) ;
    
    \path [line] (b) -- (SI21);
    \path [line] (SI21) -- (SI22);
    \path [line] (SI22) -- (keff2);
    \path [line] (keff2.east) -| ++(1.4,0cm) node [above,pos=.15] {\scriptsize no} node [pos=.3,yshift=-14pt] {\scriptsize $n=n+1$} |- ($(SI21.north)+(0,.2cm)$);
    \path [line] (keff2) -- (err) node [near start] {\scriptsize yes};
    
    \path [line] (err) -- (mesh);
    
    \path [line] (mesh.west) -- node [above,pos=.15,yshift=12pt] {\scriptsize no} (adapt);
    \path [draw] (adapt.west) to[myncbar,angle=180,arm=3.2cm] (a);
    \path [line] (mesh) -- node [above] {\scriptsize yes} (end);
    
    \begin{pgfonlayer}{background}
      \coordinate (c) at ($(SI11.west)+(-.85,.7cm)$);
      \coordinate (d) at ($(keff1.south)!.5!(ref_glob.north)+(2.25cm,0cm)$);
      \coordinate (e) at ($(SI21.west)+(-.85,.7cm)$);
      \coordinate (f) at ($(d)+(5.4cm,0cm)$);
      \coordinate (g) at ($(c)+(-.5,.4cm)$);
      \coordinate (h) at ($(f)+(.5cm,-1.15cm)$);

      % Draw the background
      \path[fill=yellow!20,rounded corners, draw=black!75, solid,opacity=1]
      (g) rectangle (h);
      \path[fill=red!20,rounded corners, draw=black!50, dashed,opacity=1]
      (c) rectangle (d);
      \path[fill=green!20,rounded corners, draw=black!50, dashed,opacity=1]
      (e) rectangle (f);
      \node [color=blue!85!black, text width=2cm, font=\scriptsize\itshape] at ($(d)+(-4.0cm,.4cm)$) {source iter.: coarse mesh};
      \node [color=blue!85!black, text width=2cm, font=\scriptsize\itshape] at ($(f)+(-3.9cm,.4cm)$) {source iter.: refined mesh};
    \end{pgfonlayer}
\end{tikzpicture}


\end{frame}
