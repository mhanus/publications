\begin{frame}[t]
	\frametitle{Nodal method -- pros and cons} 
	\vspace*{1cm}
	\begin{itemize}
		\item<1-> \dkg{Very fast, memory efficient, well-suited for parallelization}
		\item<1-> \dkg{Good for retrieving global information ($\keff$, average flux)\\[1em]}
		\item<2-> \dkr{Requires high-quality coarse-mesh data\\ (non-trivial homogenization/equivalence procedures)}	
		\item<2-> \dkr{For detailed flux distribution, most nodal methods require involved \textit{aposteriori} reconstruction}
		\item<2-> \dkr{Highly geometry-dependent}
		\item<2-> \dkr{Complicated theoretical analysis (order of accuracy, convergence)}
	\end{itemize}	
	\only<2->{
		\vspace{.15cm}\color{Black}\hrule width 5.1cm
		\begin{thebibliography}{\textwidth} 
			\beamertemplatearticlebibitems
			\bibitem[CMFD]{CMFD}
			\footnotesize{Deokjung Lee},
			\newblock \scriptsize{{C}onvergence {A}nalysis of the {C}oarse {M}esh {F}inite {D}ifference {M}ethod}
			\newblock \color{Black}Ph.D. Thesis, Purdue University, 2003      
		\end{thebibliography}		
	}
\end{frame}
