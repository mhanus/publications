\begin{frame}[c]
	\frametitle{Conclusion and final thoughts}
	 \begin{itemize}
	 	\item CMFD accelerated nodal method is a highly efficient 
	 	      solution technique for systems of (not only, but mostly) elliptic PDEs on structured meshes\pause
	 	\item It is lacking behind the adaptive FEM in terms of
	 	\begin{itemize}
	 	    \item flexibility
	 	    \item detailed spatial resolution
	 	    \item theoretical background\pause
	 	\end{itemize}
	 	\item The ideas behind it naturally appear in some FEM-dominated areas too
	 	\begin{itemize}
	 		\item methods using hierarchical bases
	 		\item domain decomposition with coarse mesh acceleration\pause
	 	\end{itemize}
	 	\item We should be able to bring the two methods even closer to each other by considering the
	 	      DG approximation on the FEM side
	 \end{itemize}
\end{frame}

\begin{frame}[c]
	\frametitle{}
	\begin{center}
	  \LARGE\emph{ Thanks for attention }\\[.5em]
	  \hrule width \textwidth\vspace{.5em}
	  \normalsize\color{structure} mhanus@kma.zcu.cz
	\end{center}
\end{frame}

